\documentclass[submission,copyright,creativecommons]{eptcs}
\providecommand{\event}{TTC 2015}

% TTC additions
\usepackage{graphicx}
\usepackage{subcaption}
\usepackage{xspace}
\usepackage{rotating,amsmath,amsfonts,amssymb}
\usepackage[T1]{fontenc}
\usepackage[utf8]{inputenc}
\usepackage{todonotes}
\usepackage{float}
\usepackage{xcolor}
\usepackage{listings}
\usepackage{framed}

% custom commands
\definecolor{negcolor}{RGB}{223,49,35}
\definecolor{newcolor}{RGB}{43,183,47}
\definecolor{delcolor}{RGB}{59,66,161}

\newcommand{\union}{\cup}
\newcommand{\intersection}{\cap}
\newcommand{\parentheses}[1]{\left(#1\right)}
\newcommand{\op}[2]{\mathrm{#1}\parentheses{#2}}
\newcommand{\figref}[1]{\autoref{fig:#1}}
\newcommand{\lstref}[1]{\autoref{lst:#1}}
\newcommand{\naturaljoin}{\bowtie}
\newcommand{\antijoin}{\, \triangleright \,}
\newcommand{\eiq}{\mbox{\textsc{EMF-IncQuery}}\@\xspace}
\newcommand{\viatra}{\mbox{\textsc{Viatra}}\@\xspace}
\newcommand{\viatratwo}{\mbox{\textsc{Viatra2}}\@\xspace}
\newcommand{\evm}{\mbox{\textsc{Viatra-EVM}}\@\xspace}
\newcommand{\jdt}{\mbox{Eclipse JDT}\@\xspace}
\newcommand{\xtend}{\mbox{Xtend}\@\xspace}
\newcommand{\iqpl}{\mbox{\textsc{IncQuery}} Pattern Language\@\xspace}
\newcommand{\tb}{Train Benchmark\@\xspace}

\renewcommand{\sectionautorefname}{Section}
\renewcommand{\subsectionautorefname}{Section}
\renewcommand{\subsubsectionautorefname}{Section}

\newcommand{\ie}{i.e.\@\xspace}
\newcommand{\Ie}{I.e.\@\xspace}
\newcommand{\eg}{e.g.\@\xspace}
\newcommand{\Eg}{E.g.\@\xspace}
\newcommand{\etal}{et al.\@\xspace}
\newcommand{\etc}{etc.\@\xspace}

\newcommand{\ttcsf}[1]{\textsf{\small #1}}
\newcommand{\ttcfig}[3]{
\begin{figure}[htb]
	\centering
	\includegraphics[width=#3\textwidth]{figures/#1}
	\caption{#2.}
	\label{fig:#1}
\end{figure}}

\newcommand{\ttcpattern}[2]{
\begin{figure}[H] 
	\centering
	\includegraphics[scale=0.3]{figures/#1}
	\caption{#2.}
	\label{fig:#1}
\end{figure}}

\newcommand{\ttcfigscale}[3]{
\begin{figure}[htb]
	\centering
	\includegraphics[scale=#3]{figures/#1}
	\caption{#2.}
	\label{fig:#1}
\end{figure}}

\newcommand{\ttcsubfig}[2]{
	\begin{subfigure}[b]{0.49\textwidth}
		\centering
		\includegraphics[scale=0.24]{figures/#1}
		\caption{#2}
		\label{fig:#1}
	\end{subfigure}
}

\definecolor{lightgray}{RGB}{242,242,242}
\definecolor{keywordcolor}{RGB}{0,0,160}
\definecolor{commentcolor}{RGB}{0,128,64}
\lstset{
	numbers=left,
	numberstyle=\scriptsize\ttfamily,
	stepnumber=1,
	numbersep=5pt,
	%
	backgroundcolor=\color{lightgray},
	basicstyle=\scriptsize\ttfamily, % print whole listing small
	keywordstyle=\color{keywordcolor}\bfseries\ttfamily,
	commentstyle=\color{commentcolor}\ttfamily,
	stringstyle=\color{stringcolor}\ttfamily,
	identifierstyle=, % nothing happens
	stringstyle=\scriptsize,
	%
	showstringspaces=false, % no special string spaces
	aboveskip=3pt,
	belowskip=3pt,
	columns=flexible,
	keepspaces=true,
	breaklines=true,	
	frameround=tttt,
	captionpos=b,
	tabsize=2,
	frame=tb,
	framerule=0pt,
	framexleftmargin=0.25em,
}

\lstdefinelanguage{iqpl}
{
	morekeywords={@QueryBasedFeature,@Constraint,count,pattern,package,neg,find,import,true,false,or,check,job,action,state,severity,location,message,oclIsKindOf,self,exists,includes,invariant,class},
	sensitive=true,
	morecomment=[l]{//},
	morecomment=[s]{/*}{*/},
	morestring=[b]{"},
}

\lstdefinelanguage{xtend}{
	morekeywords={class,public,def,val,var,cached,case,default,extension,false,import,JAVA,WORKFLOWSLOT,let,new,null,private,create,switch,this,true,reexport,around,if,then,else,context,DEFAULT_NO_UPDATE_AND_DISAPPEAR,DEFAULT,APPEARED,DISAPPEARED,UPDATED,processor},
	keywordstyle=[2]{\textbf},
	morecomment=[l]{//},
	morecomment=[s]{/*}{*/},
	morestring=[b]",
	mathescape=true,
}

\newcommand{\listingiqpl}[2]
{
	\lstset{
		language=iqpl
	}
	\lstinputlisting[label=lst:#1, caption=#2.]{code/#1}
}

\newcommand{\listingxtend}[2]{
	\lstset{
		language=xtend,
		escapeinside={(*@}{@*)},
		literate=*{[guilleft]}{\guillemotleft{}}{1}{[guilright]}{\guillemotright{}}{1}
	}
	\lstinputlisting[label=lst:#1, caption=#2.]{code/#1}
}

\newcommand{\listingjava}[2]{
	\lstset{
		language=Java
	}
	\lstinputlisting[label=lst:#1, caption=#2.]{code/#1}
}

\usepackage{placeins}
\usepackage{wrapfig}

\newcommand{\ttctransformation}[2]{
	\begin{wrapfigure}{r}{7cm}
		\centering
		\includegraphics[scale=0.22]{figures/transformation-#1}
		\vspace{-3em}
	\end{wrapfigure}
}

% metadata
\title{\tb Case: an \eiq Solution\thanks{This work was partially supported by the MONDO (EU ICT-611125) project and Red Hat Inc.}}
\author{G\'{a}bor Sz\'{a}rnyas \qquad M\'{a}rton B\'{u}r \qquad Istv\'{a}n R\'{a}th
\institute{Budapest University of Technology and Economics\\
Department of Measurement and Information Systems\\
H-1117 Magyar tud\'{o}sok krt. 2, Budapest, Hungary}
\email{szarnyas@mit.bme.hu, marton.bur@inf.mit.bme.hu, rath@mit.bme.hu}
}
\def\titlerunning{\tb Case: an \eiq Solution}
\def\authorrunning{G. Sz\'{a}rnyas et al.}

\newcommand{\plotscale}{0.32}

\begin{document}
\maketitle

\begin{abstract}
This paper presents a solution for the \tb Case of the 2015 Transformation Tool Contest, using \eiq.
\end{abstract}

\section{Introduction}

This paper describes a solution for the TTC 2015 \tb Case. The source code of the solution is available as an open-source project.\footnote{\url{https://github.com/FTSRG/trainbenchmark-ttc}} There is also a SHARE image available.\footnote{\url{http://is.ieis.tue.nl/staff/pvgorp/share/?page=ConfigureNewSession&vdi=ArchLinux64_TrainBenchmark-EIQ.vdi}}

\section{\eiq}

Automated model transformations are frequently integrated with modeling environments, requiring both high performance and a concise programming interface to support software engineers. The objective of the \eiq~\cite{models2010, eiq-homepage} framework is to provide a declarative way to define queries over EMF models. \eiq extended the pattern language of \viatratwo with new features (including transitive closure, role navigation, match count) and tailored it to EMF models~\cite{iqpl}. \eiq is developed with a focus on \emph{incremental query evaluation}, however, the most recent version is also capable of evaluating queries with a \emph{local search-based} algorithm.

\subsection{Incremental Pattern Matching}

\eiq uses the Rete algorithm~\cite{BergmannPhD} to perform incremental pattern matching. The Rete algorithm uses tuples to represent the model objects, attributes, references and partial matches in the model. The algorithm defines an asynchronous network of communicating nodes. The network consists of three types of nodes. Input nodes are responsible for indexing the model by type, \ie they store the appropriate tuples for the objects and references. They are also responsible for producing the update messages and propagating them to the worker nodes. Worker nodes perform a transformation on the output of their parent node(s) and propagate the results. Partial query results are represented in tuples and stored in the memory of the worker node, thus allowing for incremental query reevaluation. Production nodes are terminators that provide an interface for fetching the results of the query and the changes introduced by the latest transformation. Moreover, parallelization possibilities of the algorithm was already investigated in~\cite{Bergmann-gtvmt09}.

The incremental pattern matcher provides quick reevaluation for complex queries. However, it does so at the expense of high memory consumption as the partial results are stored in the Rete network.

\subsection{Local Search-Based Pattern Matching}

\emph{Local search-based pattern matching}~(LS) is commonly used in graph transformation tools~\cite{FUJABA,GrGen}. Along with the incremental query engine, \eiq also provides a local search-based pattern matcher. 

The matching process consists of four steps. ($1$) At first, in a preprocessing step the patterns are \emph{normalized}: the constraint set is minimized, variables that are always equal are unified and positive pattern calls are flattened. These normalized patterns are evaluated by ($2$) the \emph{query planner}, using a specified cost estimation function to provide search plans: %~\cite{varro:adaptivesearch}: 
totally ordered lists of search operations used to ensure that the constraints from the pattern definition hold. From a single pattern specification multiple search plans can be derived, thus pattern matching includes ($3$) \emph{plan selection} based on the input parameter binding and model-specific metrics. Finally, ($4$) \emph{the search plan is executed} by a plan interpreter evaluating the different operations of the plans. If an operation fails, the interpreter backtracks; if all operations are executed successfully, a match is found.

Compared to the incremental query engine, the search-based algorithm requires less memory~\cite{IST15} and is therefore capable of performing queries on larger models if there is not enough memory available for the incremental engine.

\subsection{Defining the Pattern Matching Strategy}

Currently, the pattern matching strategy has to be determined by the developer by specifying the query backend of \eiq. Developing a hybrid pattern matching engine is subject to future work. This will allow the user to use annotations to define the evaluation strategy for each pattern. There are also plans to develop an adaptive query engine (\autoref{summary}).

\subsection{Pattern Match Representation}

For each query, \eiq generates a set of utility classes. These classes store the model objects in the match and provide a convenient interface for reading and transforming the matches. These classes are used for implementing the transformation operations (\autoref{patterns}).

\section{Solution}

The case defines a well-formedness validation scenario set in the domain of railway systems~\cite{ttc-trainbenchmark-case}. The case provides a synthetic instance model generator which is capable of generating models of various sizes. For the solution, we used the metamodel defined in the case description without any modifications or extensions.

The solution was developed in the Eclipse IDE. For setting up the development environment, please refer to the readme file. The projects are not tied to the Eclipse environment and can be compiled with the Apache Maven~\cite{Maven} build automation tool. This offers a number of benefits, including portability and the possibility of continuous integration. The solution is written in Java~7. The patterns are defined in \iqpl (IQPL)~\cite{iqpl}.

\subsection{Example Query: \textsf{RouteSensor}}
\label{routesensor}
We describe the implementation of the \textsf{RouteSensor} query in detail. The other queries and transformations are implemented in a similar manner. The implemented application uses the Java classes generated by \eiq and the hand-coded transformation logic introduced below. First it finds the matches of the queries, then the corresponding transformation step is applied for each match. The code of patterns and the transformation definitions are listed in~\autoref{patterns}.

\ttctransformation{routesensor}

The \textsf{RouteSensor} query looks for sensors that are connected to a switch, but the sensor and the switch are not connected to the same route. The query in IQPL is listed in~\autoref{lst:RouteSensor.eiq}. The positive conditions are defined by using the appropriate classes and references, while the negative application condition (NAC) is defined as a negative find operation (\texttt{neg find}) for a separate query.

%\begin{figure}
%	\centering
%	\ttcsubfig{pattern-routesensor}{Pattern}
%	\ttcsubfig{transformation-routesensor}{Transformation}
%	\caption{Pattern and transformation of the \textsf{RouteSensor} query.}
%\end{figure}

\listingiqpl{RouteSensor.eiq}{Pattern of the \textsf{RouteSensor} query}

During the \textsf{repair} operation, for the selected matches, the missing \textsf{definedBy} edge is inserted by connecting the
route to the sensor. The Java transformation code implementing the transformation is listed in~\autoref{lst:RouteSensorTransformation.java}. The transformation uses the match object returned by \eiq.

%\ttcpattern{transformation-routesensor}{Transformation of the \textsf{RouteSensor} query}

\listingjava{RouteSensorTransformation.java}{Transformation of the \textsf{RouteSensor} query}

\subsection{Query Evaluation Strategies for the \textsf{RouteSensor} Pattern}

We use the \textsf{RouteSensor} query to provide an overview of the various query evaluation strategies used in \eiq.

\subsubsection{Incremental Evaluation}

The Rete network derived from the \textsf{RouteSensor} query is shown in~\autoref{fig:rete-routesensor-layout}. For the sake of clarity, we simplified the Rete network by removing some implementation-specific details. The evaluation in the Rete network starts with the \emph{input nodes} (\textsf{switch}, \textsf{follows}, \textsf{sensor}, \textsf{definedBy}), which are indexing the model by collecting the appropriate tuples. The \emph{worker nodes} are responsible for performing the relational operations, \emph{join} and \emph{antijoin} in this case. The join nodes have a pair of tuple masks (\eg $\langle 2, 3 \rangle$ and $\langle 0, 1 \rangle$) to determine the attributes used in the join operation. The match set of the pattern is stored in the \emph{production node}.

\subsubsection{Local Search-Based Evaluation}

The search plan generated for evaluating the \textsf{RouteSensor} query is shown in~\autoref{fig:searchplan-routesensor}. This screenshot is taken from the \emph{Local Search Debugger} view of \eiq. The search plan is presented in the upper-left part, while the found matches with the variable substitutions are shown below the search plan in a table viewer. The Zest-based graph viewer in the right visualises a match based on the selection of the table viewer.

\section{Evaluation}

In this section, we present the benchmark environment and evaluate the results. %The detailed results are shown in~\autoref{detailed-benchmark-results} (\autoref{tab:first-validation-poslength}--\autoref{tab:revalidation-switchset}).

\subsection{Benchmark Environment}

The benchmarks were performed on a 64-bit Ubuntu Server 14.04 virtual machine deployed on a private cloud. The machine used a quad-core 2.50~GHz Xeon L5420 processor and 16~GB of memory. We used Oracle JDK 8 and set the available heap memory to 15~GB.

\subsection{Benchmark Results}
\label{benchmark-results}

To present the results, we use the reporting framework of the \tb. The framework generates plots to visualise the execution time of the phases defined in the benchmark. %\footnote{The framework is also capable to visualise the memory consumption of the provided solutions. However, it is difficult to measure the memory consumption of applications running in the JVM.}
The plots showing each query are included in~\autoref{detailed-benchmark-results}. On each plot, the $x$ axis shows the problem size, \ie the size of the instance model, while the $y$ axis shows the aggregated execution time of a certain phases, measured in milliseconds. Both axes use logarithmic scale.

\subsubsection{Benchmark Results for the \textsf{RouteSensor} Query}

For the sake of conciseness, we only discuss the results for the \textsf{RouteSensor} query in detail. 

The results for the \emph{batch validation} are shown in~\autoref{fig:RouteSensor-batch-validation}. The results suggest that---given enough memory---both the incremental and the local search-based (LS) strategies are able to run the query and the transformation for the largest model. The \emph{first validation} takes consistently longer for the incremental strategy as for the LS strategy. This is caused by the fact that the incremental strategy builds the Rete algorithm during the \textsf{read} phase. However, the difference is small as the \emph{first validation} time largely consists of deserializing the EMF model.

The execution times of the \emph{revalidation} are shown in~\autoref{fig:RouteSensor-revalidation}. The execution time of the incremental strategy linearly correlates with the size of the change set. This implies that for a \emph{fixed} change set, the incremental strategy is able to perform the transformation in constant time, while execution time for the LS strategy correlates with the model size. For the \emph{proportional} change set, the revalidation time is a low-degree polynomial of the model size for both strategies, however, it is an order of magnitude faster for the incremental strategy than for the LS.

\subsection{Comparison of the Query Evaluation Strategies}

\autoref{detailed-benchmark-results} shows the detailed results for all queries and both evaluation strategies. In the \emph{first validation} (\autoref{fig:incremental-first-validation} and \autoref{fig:localsearch-first-validation}), the evaluation strategies show similar performance characteristics as both have to compute the complete result set of the query. The execution times for both strategies show that the most complex query is \textsf{SemaphoreNeighbor}, while the simplest one is \textsf{SwitchSensor}. 

As expected, the execution times of the \emph{revalidation} are different for the two strategies. \autoref{fig:incremental-revalidation} shows that for the incremental strategy the execution time correlates with the size of the match set (instead of the model size). This can be observed when comparing the execution times for the \emph{fixed} and the \emph{proportional} change sets. \autoref{fig:localsearch-revalidation} shows that the execution time for the LS strategy is determined by the model size and is not affected by the size of the change set.

These results imply that the optimal evaluation strategy depends on the specific workload profile. If there is enough memory available and the transformations operate on a small amount of model elements, it is recommended to use the incremental strategy. If the transformations often change a large proportion of the model elements, the LS strategy is recommended.

\section{Summary and Future Work}
\label{summary}

The paper presented a solution for the \tb case of the 2015 Transformation Tool Contest. 

There is ongoing work to develop a hybrid query engine~\cite{STTT10} for \eiq. This will allow the user to use annotations on the patterns for specifying the desired query evaluation strategy. There are also plans to develop an adaptive query engine which will use query optimisation heuristics to determine the appropriate strategy based on the query, the model and the available resources.

\paragraph{Acknowledgements.} The authors would like to thank Zoltán Ujhelyi for providing valuable insights into \eiq.

\bibliographystyle{eptcs}
\bibliography{ttc}

\clearpage

\appendix
\section{Appendix}

\subsection{Patterns and Transformations}
\label{patterns}

\subsubsection{\textsf{PosLength}}

\listingiqpl{PosLength.eiq}{Pattern of the \textsf{PosLength} query}

\listingjava{PosLengthTransformation.java}{Transformation of the \textsf{PosLength} query}

\subsubsection{\textsf{SwitchSensor}}

\listingiqpl{SwitchSensor.eiq}{Pattern of the \textsf{SwitchSensor} query}

\listingjava{SwitchSensorTransformation.java}{Transformation of the \textsf{SwitchSensor} query}

\subsubsection{\textsf{SwitchSet}}

\listingiqpl{SwitchSet.eiq}{Pattern of the \textsf{SwitchSet} query}

\listingjava{SwitchSetTransformation.java}{Transformation of the \textsf{SwitchSet} query}

\subsubsection{\textsf{RouteSensor}}

The \textsf{RouteSensor} query is discussed in detail in \autoref{routesensor}.

%\listingiqpl{RouteSensor.eiq}{Pattern of the \textsf{RouteSensor} query}

%\listingjava{RouteSensorTransformation.java}{Transformation of the \textsf{RouteSensor} query}

\subsubsection{\textsf{SemaphoreNeighbor}}

\listingiqpl{SemaphoreNeighbor.eiq}{Pattern of the \textsf{SemaphoreNeighbor} query}

\listingjava{SemaphoreNeighborTransformation.java}{Transformation of the \textsf{SemaphoreNeighbor} query}

\clearpage

\subsection{Detailed Benchmark Results}
\label{detailed-benchmark-results}

\begin{figure}[h!]
	\centering
	\ttcsubfig{fixed-RouteSensor-GroupBy-Tool-time-batch-validation}{Fixed change set}
	\ttcsubfig{proportional-RouteSensor-GroupBy-Tool-time-batch-validation}{Proportional change set}
	\caption{First validation times for the \textsf{RouteSensor} query.}
	\label{fig:RouteSensor-batch-validation}
\end{figure}

\begin{figure}[h!]
	\centering
	\ttcsubfig{fixed-RouteSensor-GroupBy-Tool-time-revalidation}{Fixed change set}
	\ttcsubfig{proportional-RouteSensor-GroupBy-Tool-time-revalidation}{Proportional change set}
	\caption{Revalidation times for the \textsf{RouteSensor} query.}
	\label{fig:RouteSensor-revalidation}
\end{figure}

\begin{figure}[h!]
	\centering
	\ttcsubfig{fixed-EMFIncQuery-Incremental-GroupBy-Query-time-batch-validation}{Fixed change set}
	\ttcsubfig{proportional-EMFIncQuery-Incremental-GroupBy-Query-time-batch-validation}{Proportional change set}
	\caption{First validation times for the \emph{incremental} query evaluation strategy.}
	\label{fig:incremental-first-validation}
\end{figure}

\begin{figure}[h!]
	\centering
	\ttcsubfig{fixed-EMFIncQuery-Incremental-GroupBy-Query-time-revalidation}{Fixed change set}
	\ttcsubfig{proportional-EMFIncQuery-Incremental-GroupBy-Query-time-revalidation}{Proportional change set}
	\caption{Revalidation times for the \emph{incremental} query evaluation strategy.}
	\label{fig:incremental-revalidation}
\end{figure}

\begin{figure}[h!]
	\centering
	\ttcsubfig{fixed-EMFIncQuery-LocalSearch-GroupBy-Query-time-batch-validation}{Fixed change set}
	\ttcsubfig{proportional-EMFIncQuery-LocalSearch-GroupBy-Query-time-batch-validation}{Proportional change set}
	\caption{First validation times for the \emph{local search-based} query evaluation strategy.}
	\label{fig:localsearch-first-validation}
\end{figure}

\begin{figure}[h!]
	\centering
	\ttcsubfig{fixed-EMFIncQuery-LocalSearch-GroupBy-Query-time-revalidation}{Fixed change set}
	\ttcsubfig{proportional-EMFIncQuery-LocalSearch-GroupBy-Query-time-revalidation}{Proportional change set}
	\caption{Revalidation times for the \emph{local search-based} query evaluation strategy.}
	\label{fig:localsearch-revalidation}
\end{figure}

% tables with the benchmark results

%\FloatBarrier

%\begin{table}
\centering
\footnotesize
\begin{tabular}{| l | l | r | l | l | l | r |}
\hline
\sf ChangeSet & \sf Tool & \sf Size & \sf Query & \sf PhaseName & \sf MetricName & \sf MetricValue\\\hline
fixed & EMFIncQuery-Incremental & 1 & PosLength & read and check & time & 210.298071\\\hline
fixed & EMFIncQuery-LocalSearch & 1 & PosLength & read and check & time & 178.194132\\\hline
fixed & EMFIncQuery-Incremental & 2 & PosLength & read and check & time & 205.277574\\\hline
fixed & EMFIncQuery-LocalSearch & 2 & PosLength & read and check & time & 191.203654\\\hline
fixed & EMFIncQuery-Incremental & 4 & PosLength & read and check & time & 343.749371\\\hline
fixed & EMFIncQuery-LocalSearch & 4 & PosLength & read and check & time & 200.046779\\\hline
fixed & EMFIncQuery-Incremental & 8 & PosLength & read and check & time & 452.233927\\\hline
fixed & EMFIncQuery-LocalSearch & 8 & PosLength & read and check & time & 212.041935\\\hline
fixed & EMFIncQuery-Incremental & 16 & PosLength & read and check & time & 410.4975\\\hline
fixed & EMFIncQuery-LocalSearch & 16 & PosLength & read and check & time & 367.167569\\\hline
fixed & EMFIncQuery-Incremental & 32 & PosLength & read and check & time & 801.159519\\\hline
fixed & EMFIncQuery-LocalSearch & 32 & PosLength & read and check & time & 796.58343\\\hline
fixed & EMFIncQuery-Incremental & 64 & PosLength & read and check & time & 1424.413731\\\hline
fixed & EMFIncQuery-LocalSearch & 64 & PosLength & read and check & time & 1528.145427\\\hline
fixed & EMFIncQuery-Incremental & 128 & PosLength & read and check & time & 2826.950307\\\hline
fixed & EMFIncQuery-LocalSearch & 128 & PosLength & read and check & time & 3219.605669\\\hline
fixed & EMFIncQuery-Incremental & 256 & PosLength & read and check & time & 5706.533909\\\hline
fixed & EMFIncQuery-LocalSearch & 256 & PosLength & read and check & time & 5880.109566\\\hline
fixed & EMFIncQuery-Incremental & 512 & PosLength & read and check & time & 11232.23632\\\hline
fixed & EMFIncQuery-LocalSearch & 512 & PosLength & read and check & time & 11863.238066\\\hline
fixed & EMFIncQuery-Incremental & 1024 & PosLength & read and check & time & 22894.174829\\\hline
fixed & EMFIncQuery-LocalSearch & 1024 & PosLength & read and check & time & 24157.24701\\\hline
fixed & EMFIncQuery-Incremental & 2048 & PosLength & read and check & time & 44280.148132\\\hline
fixed & EMFIncQuery-LocalSearch & 2048 & PosLength & read and check & time & 48044.002399\\\hline
fixed & EMFIncQuery-Incremental & 4096 & PosLength & read and check & time & 86182.94385\\\hline
fixed & EMFIncQuery-LocalSearch & 4096 & PosLength & read and check & time & 90882.464242\\\hline
fixed & EMFIncQuery-Incremental & 8192 & PosLength & read and check & time & 225517.528734\\\hline
fixed & EMFIncQuery-LocalSearch & 8192 & PosLength & read and check & time & 231307.538858\\\hline

proportional & EMFIncQuery-Incremental & 1 & PosLength & read and check & time & 224.381759\\\hline
proportional & EMFIncQuery-LocalSearch & 1 & PosLength & read and check & time & 178.372412\\\hline
proportional & EMFIncQuery-Incremental & 2 & PosLength & read and check & time & 229.628186\\\hline
proportional & EMFIncQuery-LocalSearch & 2 & PosLength & read and check & time & 188.699673\\\hline
proportional & EMFIncQuery-Incremental & 4 & PosLength & read and check & time & 344.922543\\\hline
proportional & EMFIncQuery-LocalSearch & 4 & PosLength & read and check & time & 201.548687\\\hline
proportional & EMFIncQuery-Incremental & 8 & PosLength & read and check & time & 363.863428\\\hline
proportional & EMFIncQuery-LocalSearch & 8 & PosLength & read and check & time & 215.223526\\\hline
proportional & EMFIncQuery-Incremental & 16 & PosLength & read and check & time & 367.06856\\\hline
proportional & EMFIncQuery-LocalSearch & 16 & PosLength & read and check & time & 360.902505\\\hline
proportional & EMFIncQuery-Incremental & 32 & PosLength & read and check & time & 776.901558\\\hline
proportional & EMFIncQuery-LocalSearch & 32 & PosLength & read and check & time & 808.378187\\\hline
proportional & EMFIncQuery-Incremental & 64 & PosLength & read and check & time & 1426.435449\\\hline
proportional & EMFIncQuery-LocalSearch & 64 & PosLength & read and check & time & 1519.517305\\\hline
proportional & EMFIncQuery-Incremental & 128 & PosLength & read and check & time & 2828.249139\\\hline
proportional & EMFIncQuery-LocalSearch & 128 & PosLength & read and check & time & 3057.928077\\\hline
proportional & EMFIncQuery-Incremental & 256 & PosLength & read and check & time & 5751.735546\\\hline
proportional & EMFIncQuery-LocalSearch & 256 & PosLength & read and check & time & 5883.870936\\\hline
proportional & EMFIncQuery-Incremental & 512 & PosLength & read and check & time & 12236.596997\\\hline
proportional & EMFIncQuery-LocalSearch & 512 & PosLength & read and check & time & 11843.692537\\\hline
proportional & EMFIncQuery-Incremental & 1024 & PosLength & read and check & time & 23191.679772\\\hline
proportional & EMFIncQuery-LocalSearch & 1024 & PosLength & read and check & time & 24560.663609\\\hline
proportional & EMFIncQuery-Incremental & 2048 & PosLength & read and check & time & 47247.65474\\\hline
proportional & EMFIncQuery-LocalSearch & 2048 & PosLength & read and check & time & 45466.508926\\\hline
proportional & EMFIncQuery-Incremental & 4096 & PosLength & read and check & time & 87092.486941\\\hline
proportional & EMFIncQuery-LocalSearch & 4096 & PosLength & read and check & time & 90897.766231\\\hline
proportional & EMFIncQuery-Incremental & 8192 & PosLength & read and check & time & 220624.693392\\\hline
proportional & EMFIncQuery-LocalSearch & 8192 & PosLength & read and check & time & 233514.346171\\\hline

\end{tabular}\caption{Benchmark results for the \textsf{PosLength} query.}
\label{tab:first-validation-poslength}
\end{table}

%\begin{table}
\centering
\footnotesize
\begin{tabular}{| l | l | r | l | l | l | r |}
\hline
\sf ChangeSet & \sf Tool & \sf Size & \sf Query & \sf PhaseName & \sf MetricName & \sf MetricValue\\\hline

fixed & EMFIncQuery-Incremental & 1 & PosLength & modify and revalidate & time & 6.103262\\\hline
fixed & EMFIncQuery-LocalSearch & 1 & PosLength & modify and revalidate & time & 18.049115\\\hline
fixed & EMFIncQuery-Incremental & 2 & PosLength & modify and revalidate & time & 7.449719\\\hline
fixed & EMFIncQuery-LocalSearch & 2 & PosLength & modify and revalidate & time & 31.779216\\\hline
fixed & EMFIncQuery-Incremental & 4 & PosLength & modify and revalidate & time & 8.624596\\\hline
fixed & EMFIncQuery-LocalSearch & 4 & PosLength & modify and revalidate & time & 78.423165\\\hline
fixed & EMFIncQuery-Incremental & 8 & PosLength & modify and revalidate & time & 6.058546\\\hline
fixed & EMFIncQuery-LocalSearch & 8 & PosLength & modify and revalidate & time & 86.905453\\\hline
fixed & EMFIncQuery-Incremental & 16 & PosLength & modify and revalidate & time & 4.830845\\\hline
fixed & EMFIncQuery-LocalSearch & 16 & PosLength & modify and revalidate & time & 170.961208\\\hline
fixed & EMFIncQuery-Incremental & 32 & PosLength & modify and revalidate & time & 3.328704\\\hline
fixed & EMFIncQuery-LocalSearch & 32 & PosLength & modify and revalidate & time & 393.40069\\\hline
fixed & EMFIncQuery-Incremental & 64 & PosLength & modify and revalidate & time & 3.763323\\\hline
fixed & EMFIncQuery-LocalSearch & 64 & PosLength & modify and revalidate & time & 777.62781\\\hline
fixed & EMFIncQuery-Incremental & 128 & PosLength & modify and revalidate & time & 3.362133\\\hline
fixed & EMFIncQuery-LocalSearch & 128 & PosLength & modify and revalidate & time & 1555.411909\\\hline
fixed & EMFIncQuery-Incremental & 256 & PosLength & modify and revalidate & time & 3.540044\\\hline
fixed & EMFIncQuery-LocalSearch & 256 & PosLength & modify and revalidate & time & 2771.568631\\\hline
fixed & EMFIncQuery-Incremental & 512 & PosLength & modify and revalidate & time & 4.481992\\\hline
fixed & EMFIncQuery-LocalSearch & 512 & PosLength & modify and revalidate & time & 5475.972205\\\hline
fixed & EMFIncQuery-Incremental & 1024 & PosLength & modify and revalidate & time & 4.772878\\\hline
fixed & EMFIncQuery-LocalSearch & 1024 & PosLength & modify and revalidate & time & 10825.021794\\\hline
fixed & EMFIncQuery-Incremental & 2048 & PosLength & modify and revalidate & time & 5.279539\\\hline
fixed & EMFIncQuery-LocalSearch & 2048 & PosLength & modify and revalidate & time & 20035.255302\\\hline
fixed & EMFIncQuery-Incremental & 4096 & PosLength & modify and revalidate & time & 4.989766\\\hline
fixed & EMFIncQuery-LocalSearch & 4096 & PosLength & modify and revalidate & time & 41899.598964\\\hline
fixed & EMFIncQuery-Incremental & 8192 & PosLength & modify and revalidate & time & 6.568381\\\hline
fixed & EMFIncQuery-LocalSearch & 8192 & PosLength & modify and revalidate & time & 89270.16989\\\hline

proportional & EMFIncQuery-Incremental & 1 & PosLength & modify and revalidate & time & 3.623648\\\hline
proportional & EMFIncQuery-LocalSearch & 1 & PosLength & modify and revalidate & time & 18.703936\\\hline
proportional & EMFIncQuery-Incremental & 2 & PosLength & modify and revalidate & time & 5.890034\\\hline
proportional & EMFIncQuery-LocalSearch & 2 & PosLength & modify and revalidate & time & 33.095907\\\hline
proportional & EMFIncQuery-Incremental & 4 & PosLength & modify and revalidate & time & 14.955579\\\hline
proportional & EMFIncQuery-LocalSearch & 4 & PosLength & modify and revalidate & time & 81.266841\\\hline
proportional & EMFIncQuery-Incremental & 8 & PosLength & modify and revalidate & time & 17.729648\\\hline
proportional & EMFIncQuery-LocalSearch & 8 & PosLength & modify and revalidate & time & 98.064957\\\hline
proportional & EMFIncQuery-Incremental & 16 & PosLength & modify and revalidate & time & 24.467171\\\hline
proportional & EMFIncQuery-LocalSearch & 16 & PosLength & modify and revalidate & time & 173.529681\\\hline
proportional & EMFIncQuery-Incremental & 32 & PosLength & modify and revalidate & time & 34.138667\\\hline
proportional & EMFIncQuery-LocalSearch & 32 & PosLength & modify and revalidate & time & 399.977507\\\hline
proportional & EMFIncQuery-Incremental & 64 & PosLength & modify and revalidate & time & 40.857772\\\hline
proportional & EMFIncQuery-LocalSearch & 64 & PosLength & modify and revalidate & time & 773.646625\\\hline
proportional & EMFIncQuery-Incremental & 128 & PosLength & modify and revalidate & time & 27.551672\\\hline
proportional & EMFIncQuery-LocalSearch & 128 & PosLength & modify and revalidate & time & 1547.53811\\\hline
proportional & EMFIncQuery-Incremental & 256 & PosLength & modify and revalidate & time & 55.821364\\\hline
proportional & EMFIncQuery-LocalSearch & 256 & PosLength & modify and revalidate & time & 2651.152403\\\hline
proportional & EMFIncQuery-Incremental & 512 & PosLength & modify and revalidate & time & 104.735541\\\hline
proportional & EMFIncQuery-LocalSearch & 512 & PosLength & modify and revalidate & time & 5401.067104\\\hline
proportional & EMFIncQuery-Incremental & 1024 & PosLength & modify and revalidate & time & 213.189839\\\hline
proportional & EMFIncQuery-LocalSearch & 1024 & PosLength & modify and revalidate & time & 10644.245827\\\hline
proportional & EMFIncQuery-Incremental & 2048 & PosLength & modify and revalidate & time & 437.755818\\\hline
proportional & EMFIncQuery-LocalSearch & 2048 & PosLength & modify and revalidate & time & 18657.528983\\\hline
proportional & EMFIncQuery-Incremental & 4096 & PosLength & modify and revalidate & time & 917.094651\\\hline
proportional & EMFIncQuery-LocalSearch & 4096 & PosLength & modify and revalidate & time & 38830.112617\\\hline
proportional & EMFIncQuery-Incremental & 8192 & PosLength & modify and revalidate & time & 2172.682751\\\hline
proportional & EMFIncQuery-LocalSearch & 8192 & PosLength & modify and revalidate & time & 89303.107023\\\hline

\end{tabular}\caption{Benchmark results for the \textsf{PosLength} query.}
\label{tab:revalidation-poslength}
\end{table}


%\begin{table}
\centering
\footnotesize
\begin{tabular}{| l | l | r | l | l | l | r |}
\hline
\sf ChangeSet & \sf Tool & \sf Size & \sf Query & \sf PhaseName & \sf MetricName & \sf MetricValue\\\hline

fixed & EMFIncQuery-Incremental & 1 & RouteSensor & read and check & time & 215.721866\\\hline
fixed & EMFIncQuery-LocalSearch & 1 & RouteSensor & read and check & time & 175.855399\\\hline
fixed & EMFIncQuery-Incremental & 2 & RouteSensor & read and check & time & 227.193428\\\hline
fixed & EMFIncQuery-LocalSearch & 2 & RouteSensor & read and check & time & 185.02548\\\hline
fixed & EMFIncQuery-Incremental & 4 & RouteSensor & read and check & time & 431.596044\\\hline
fixed & EMFIncQuery-LocalSearch & 4 & RouteSensor & read and check & time & 224.715424\\\hline
fixed & EMFIncQuery-Incremental & 8 & RouteSensor & read and check & time & 436.545796\\\hline
fixed & EMFIncQuery-LocalSearch & 8 & RouteSensor & read and check & time & 201.046293\\\hline
fixed & EMFIncQuery-Incremental & 16 & RouteSensor & read and check & time & 371.533001\\\hline
fixed & EMFIncQuery-LocalSearch & 16 & RouteSensor & read and check & time & 333.507682\\\hline
fixed & EMFIncQuery-Incremental & 32 & RouteSensor & read and check & time & 789.506427\\\hline
fixed & EMFIncQuery-LocalSearch & 32 & RouteSensor & read and check & time & 731.736499\\\hline
fixed & EMFIncQuery-Incremental & 64 & RouteSensor & read and check & time & 1485.73648\\\hline
fixed & EMFIncQuery-LocalSearch & 64 & RouteSensor & read and check & time & 1345.977827\\\hline
fixed & EMFIncQuery-Incremental & 128 & RouteSensor & read and check & time & 3014.091669\\\hline
fixed & EMFIncQuery-LocalSearch & 128 & RouteSensor & read and check & time & 2649.252062\\\hline
fixed & EMFIncQuery-Incremental & 256 & RouteSensor & read and check & time & 6338.0945\\\hline
fixed & EMFIncQuery-LocalSearch & 256 & RouteSensor & read and check & time & 5292.011608\\\hline
fixed & EMFIncQuery-Incremental & 512 & RouteSensor & read and check & time & 11728.671796\\\hline
fixed & EMFIncQuery-LocalSearch & 512 & RouteSensor & read and check & time & 10901.159666\\\hline
fixed & EMFIncQuery-Incremental & 1024 & RouteSensor & read and check & time & 27821.992056\\\hline
fixed & EMFIncQuery-LocalSearch & 1024 & RouteSensor & read and check & time & 21982.096913\\\hline
fixed & EMFIncQuery-Incremental & 2048 & RouteSensor & read and check & time & 52397.700303\\\hline
fixed & EMFIncQuery-LocalSearch & 2048 & RouteSensor & read and check & time & 42075.807487\\\hline
fixed & EMFIncQuery-Incremental & 4096 & RouteSensor & read and check & time & 120394.147537\\\hline
fixed & EMFIncQuery-LocalSearch & 4096 & RouteSensor & read and check & time & 77697.378225\\\hline
fixed & EMFIncQuery-Incremental & 8192 & RouteSensor & read and check & time & 248543.496509\\\hline
fixed & EMFIncQuery-LocalSearch & 8192 & RouteSensor & read and check & time & 207841.733247\\\hline

proportional & EMFIncQuery-Incremental & 1 & RouteSensor & read and check & time & 222.867966\\\hline
proportional & EMFIncQuery-LocalSearch & 1 & RouteSensor & read and check & time & 182.249332\\\hline
proportional & EMFIncQuery-Incremental & 2 & RouteSensor & read and check & time & 351.933322\\\hline
proportional & EMFIncQuery-LocalSearch & 2 & RouteSensor & read and check & time & 188.291042\\\hline
proportional & EMFIncQuery-Incremental & 4 & RouteSensor & read and check & time & 499.211892\\\hline
proportional & EMFIncQuery-LocalSearch & 4 & RouteSensor & read and check & time & 205.069178\\\hline
proportional & EMFIncQuery-Incremental & 8 & RouteSensor & read and check & time & 367.802443\\\hline
proportional & EMFIncQuery-LocalSearch & 8 & RouteSensor & read and check & time & 200.554519\\\hline
proportional & EMFIncQuery-Incremental & 16 & RouteSensor & read and check & time & 362.626905\\\hline
proportional & EMFIncQuery-LocalSearch & 16 & RouteSensor & read and check & time & 361.76396\\\hline
proportional & EMFIncQuery-Incremental & 32 & RouteSensor & read and check & time & 806.491356\\\hline
proportional & EMFIncQuery-LocalSearch & 32 & RouteSensor & read and check & time & 735.633516\\\hline
proportional & EMFIncQuery-Incremental & 64 & RouteSensor & read and check & time & 1493.348463\\\hline
proportional & EMFIncQuery-LocalSearch & 64 & RouteSensor & read and check & time & 1348.179156\\\hline
proportional & EMFIncQuery-Incremental & 128 & RouteSensor & read and check & time & 3000.006355\\\hline
proportional & EMFIncQuery-LocalSearch & 128 & RouteSensor & read and check & time & 2689.06461\\\hline
proportional & EMFIncQuery-Incremental & 256 & RouteSensor & read and check & time & 6123.016756\\\hline
proportional & EMFIncQuery-LocalSearch & 256 & RouteSensor & read and check & time & 5361.599056\\\hline
proportional & EMFIncQuery-Incremental & 512 & RouteSensor & read and check & time & 12149.338585\\\hline
proportional & EMFIncQuery-LocalSearch & 512 & RouteSensor & read and check & time & 10928.122561\\\hline
proportional & EMFIncQuery-Incremental & 1024 & RouteSensor & read and check & time & 27387.193941\\\hline
proportional & EMFIncQuery-LocalSearch & 1024 & RouteSensor & read and check & time & 22263.203791\\\hline
proportional & EMFIncQuery-Incremental & 2048 & RouteSensor & read and check & time & 55600.828061\\\hline
proportional & EMFIncQuery-LocalSearch & 2048 & RouteSensor & read and check & time & 43194.689168\\\hline
proportional & EMFIncQuery-Incremental & 4096 & RouteSensor & read and check & time & 121172.728516\\\hline
proportional & EMFIncQuery-LocalSearch & 4096 & RouteSensor & read and check & time & 77690.516919\\\hline
proportional & EMFIncQuery-Incremental & 8192 & RouteSensor & read and check & time & 245898.560203\\\hline
proportional & EMFIncQuery-LocalSearch & 8192 & RouteSensor & read and check & time & 204455.993189\\\hline

\end{tabular}\caption{Benchmark results for the \textsf{RouteSensor} query.}
\label{tab:first-validation-routesensor}
\end{table}

%\begin{table}
\centering
\footnotesize
\begin{tabular}{| l | l | r | l | l | l | r |}
\hline
\sf ChangeSet & \sf Tool & \sf Size & \sf Query & \sf PhaseName & \sf MetricName & \sf MetricValue\\\hline

fixed & EMFIncQuery-Incremental & 1 & RouteSensor & modify and revalidate & time & 8.260157\\\hline
fixed & EMFIncQuery-LocalSearch & 1 & RouteSensor & modify and revalidate & time & 43.253546\\\hline
fixed & EMFIncQuery-Incremental & 2 & RouteSensor & modify and revalidate & time & 4.793492\\\hline
fixed & EMFIncQuery-LocalSearch & 2 & RouteSensor & modify and revalidate & time & 71.546014\\\hline
fixed & EMFIncQuery-Incremental & 4 & RouteSensor & modify and revalidate & time & 9.520789\\\hline
fixed & EMFIncQuery-LocalSearch & 4 & RouteSensor & modify and revalidate & time & 87.314068\\\hline
fixed & EMFIncQuery-Incremental & 8 & RouteSensor & modify and revalidate & time & 10.894073\\\hline
fixed & EMFIncQuery-LocalSearch & 8 & RouteSensor & modify and revalidate & time & 34.837744\\\hline
fixed & EMFIncQuery-Incremental & 16 & RouteSensor & modify and revalidate & time & 14.441809\\\hline
fixed & EMFIncQuery-LocalSearch & 16 & RouteSensor & modify and revalidate & time & 49.340817\\\hline
fixed & EMFIncQuery-Incremental & 32 & RouteSensor & modify and revalidate & time & 21.849251\\\hline
fixed & EMFIncQuery-LocalSearch & 32 & RouteSensor & modify and revalidate & time & 82.257334\\\hline
fixed & EMFIncQuery-Incremental & 64 & RouteSensor & modify and revalidate & time & 22.774575\\\hline
fixed & EMFIncQuery-LocalSearch & 64 & RouteSensor & modify and revalidate & time & 142.119137\\\hline
fixed & EMFIncQuery-Incremental & 128 & RouteSensor & modify and revalidate & time & 21.31011\\\hline
fixed & EMFIncQuery-LocalSearch & 128 & RouteSensor & modify and revalidate & time & 217.700792\\\hline
fixed & EMFIncQuery-Incremental & 256 & RouteSensor & modify and revalidate & time & 22.960783\\\hline
fixed & EMFIncQuery-LocalSearch & 256 & RouteSensor & modify and revalidate & time & 403.424013\\\hline
fixed & EMFIncQuery-Incremental & 512 & RouteSensor & modify and revalidate & time & 24.649783\\\hline
fixed & EMFIncQuery-LocalSearch & 512 & RouteSensor & modify and revalidate & time & 700.821656\\\hline
fixed & EMFIncQuery-Incremental & 1024 & RouteSensor & modify and revalidate & time & 24.943514\\\hline
fixed & EMFIncQuery-LocalSearch & 1024 & RouteSensor & modify and revalidate & time & 1344.97208\\\hline
fixed & EMFIncQuery-Incremental & 2048 & RouteSensor & modify and revalidate & time & 28.683686\\\hline
fixed & EMFIncQuery-LocalSearch & 2048 & RouteSensor & modify and revalidate & time & 2524.912137\\\hline
fixed & EMFIncQuery-Incremental & 4096 & RouteSensor & modify and revalidate & time & 33.167976\\\hline
fixed & EMFIncQuery-LocalSearch & 4096 & RouteSensor & modify and revalidate & time & 5911.011832\\\hline
fixed & EMFIncQuery-Incremental & 8192 & RouteSensor & modify and revalidate & time & 44.697337\\\hline
fixed & EMFIncQuery-LocalSearch & 8192 & RouteSensor & modify and revalidate & time & 10458.141634\\\hline


proportional & EMFIncQuery-Incremental & 1 & RouteSensor & modify and revalidate & time & 0.040294\\\hline
proportional & EMFIncQuery-LocalSearch & 1 & RouteSensor & modify and revalidate & time & 43.602038\\\hline
proportional & EMFIncQuery-Incremental & 2 & RouteSensor & modify and revalidate & time & 0.035133\\\hline
proportional & EMFIncQuery-LocalSearch & 2 & RouteSensor & modify and revalidate & time & 69.979906\\\hline
proportional & EMFIncQuery-Incremental & 4 & RouteSensor & modify and revalidate & time & 5.560636\\\hline
proportional & EMFIncQuery-LocalSearch & 4 & RouteSensor & modify and revalidate & time & 87.452915\\\hline
proportional & EMFIncQuery-Incremental & 8 & RouteSensor & modify and revalidate & time & 7.666914\\\hline
proportional & EMFIncQuery-LocalSearch & 8 & RouteSensor & modify and revalidate & time & 37.851678\\\hline
proportional & EMFIncQuery-Incremental & 16 & RouteSensor & modify and revalidate & time & 8.908098\\\hline
proportional & EMFIncQuery-LocalSearch & 16 & RouteSensor & modify and revalidate & time & 47.113066\\\hline
proportional & EMFIncQuery-Incremental & 32 & RouteSensor & modify and revalidate & time & 20.372224\\\hline
proportional & EMFIncQuery-LocalSearch & 32 & RouteSensor & modify and revalidate & time & 79.665794\\\hline
proportional & EMFIncQuery-Incremental & 64 & RouteSensor & modify and revalidate & time & 42.722508\\\hline
proportional & EMFIncQuery-LocalSearch & 64 & RouteSensor & modify and revalidate & time & 143.841329\\\hline
proportional & EMFIncQuery-Incremental & 128 & RouteSensor & modify and revalidate & time & 66.745304\\\hline
proportional & EMFIncQuery-LocalSearch & 128 & RouteSensor & modify and revalidate & time & 241.419406\\\hline
proportional & EMFIncQuery-Incremental & 256 & RouteSensor & modify and revalidate & time & 79.390471\\\hline
proportional & EMFIncQuery-LocalSearch & 256 & RouteSensor & modify and revalidate & time & 414.688406\\\hline
proportional & EMFIncQuery-Incremental & 512 & RouteSensor & modify and revalidate & time & 123.654424\\\hline
proportional & EMFIncQuery-LocalSearch & 512 & RouteSensor & modify and revalidate & time & 744.099881\\\hline
proportional & EMFIncQuery-Incremental & 1024 & RouteSensor & modify and revalidate & time & 245.683569\\\hline
proportional & EMFIncQuery-LocalSearch & 1024 & RouteSensor & modify and revalidate & time & 1481.35725\\\hline
proportional & EMFIncQuery-Incremental & 2048 & RouteSensor & modify and revalidate & time & 593.285299\\\hline
proportional & EMFIncQuery-LocalSearch & 2048 & RouteSensor & modify and revalidate & time & 2854.712723\\\hline
proportional & EMFIncQuery-Incremental & 4096 & RouteSensor & modify and revalidate & time & 1614.758103\\\hline
proportional & EMFIncQuery-LocalSearch & 4096 & RouteSensor & modify and revalidate & time & 7125.842407\\\hline
proportional & EMFIncQuery-Incremental & 8192 & RouteSensor & modify and revalidate & time & 5427.66646\\\hline
proportional & EMFIncQuery-LocalSearch & 8192 & RouteSensor & modify and revalidate & time & 14897.012763\\\hline

\end{tabular}\caption{Benchmark results for the \textsf{RouteSensor} query.}
\label{tab:revalidation-routesensor}
\end{table}


%\begin{table}
\centering
\footnotesize
\begin{tabular}{| l | l | r | l | l | l | r |}
\hline
\sf ChangeSet & \sf Tool & \sf Size & \sf Query & \sf PhaseName & \sf MetricName & \sf MetricValue\\\hline

fixed & EMFIncQuery-Incremental & 1 & SemaphoreNeighbor & read and check & time & 240.232289\\\hline
fixed & EMFIncQuery-LocalSearch & 1 & SemaphoreNeighbor & read and check & time & 182.510278\\\hline
fixed & EMFIncQuery-Incremental & 2 & SemaphoreNeighbor & read and check & time & 227.46587\\\hline
fixed & EMFIncQuery-LocalSearch & 2 & SemaphoreNeighbor & read and check & time & 228.41688\\\hline
fixed & EMFIncQuery-Incremental & 4 & SemaphoreNeighbor & read and check & time & 386.769898\\\hline
fixed & EMFIncQuery-LocalSearch & 4 & SemaphoreNeighbor & read and check & time & 228.362674\\\hline
fixed & EMFIncQuery-Incremental & 8 & SemaphoreNeighbor & read and check & time & 295.670713\\\hline
fixed & EMFIncQuery-LocalSearch & 8 & SemaphoreNeighbor & read and check & time & 249.644801\\\hline
fixed & EMFIncQuery-Incremental & 16 & SemaphoreNeighbor & read and check & time & 442.122157\\\hline
fixed & EMFIncQuery-LocalSearch & 16 & SemaphoreNeighbor & read and check & time & 413.637163\\\hline
fixed & EMFIncQuery-Incremental & 32 & SemaphoreNeighbor & read and check & time & 1013.83655\\\hline
fixed & EMFIncQuery-LocalSearch & 32 & SemaphoreNeighbor & read and check & time & 927.578515\\\hline
fixed & EMFIncQuery-Incremental & 64 & SemaphoreNeighbor & read and check & time & 1925.260776\\\hline
fixed & EMFIncQuery-LocalSearch & 64 & SemaphoreNeighbor & read and check & time & 1643.771442\\\hline
fixed & EMFIncQuery-Incremental & 128 & SemaphoreNeighbor & read and check & time & 4446.306889\\\hline
fixed & EMFIncQuery-LocalSearch & 128 & SemaphoreNeighbor & read and check & time & 3400.386332\\\hline
fixed & EMFIncQuery-Incremental & 256 & SemaphoreNeighbor & read and check & time & 7856.981111\\\hline
fixed & EMFIncQuery-LocalSearch & 256 & SemaphoreNeighbor & read and check & time & 6416.952077\\\hline
fixed & EMFIncQuery-Incremental & 512 & SemaphoreNeighbor & read and check & time & 15988.99645\\\hline
fixed & EMFIncQuery-LocalSearch & 512 & SemaphoreNeighbor & read and check & time & 13022.04044\\\hline
fixed & EMFIncQuery-Incremental & 1024 & SemaphoreNeighbor & read and check & time & 40866.085543\\\hline
fixed & EMFIncQuery-LocalSearch & 1024 & SemaphoreNeighbor & read and check & time & 25767.134546\\\hline
fixed & EMFIncQuery-Incremental & 2048 & SemaphoreNeighbor & read and check & time & 105934.257489\\\hline
fixed & EMFIncQuery-LocalSearch & 2048 & SemaphoreNeighbor & read and check & time & 52355.644842\\\hline
fixed & EMFIncQuery-Incremental & 4096 & SemaphoreNeighbor & read and check & time & 239544.128485\\\hline
fixed & EMFIncQuery-LocalSearch & 4096 & SemaphoreNeighbor & read and check & time & 107863.641072\\\hline
fixed & EMFIncQuery-LocalSearch & 8192 & SemaphoreNeighbor & read and check & time & 270211.180376\\\hline

proportional & EMFIncQuery-Incremental & 1 & SemaphoreNeighbor & read and check & time & 221.744701\\\hline
proportional & EMFIncQuery-LocalSearch & 1 & SemaphoreNeighbor & read and check & time & 189.005514\\\hline
proportional & EMFIncQuery-Incremental & 2 & SemaphoreNeighbor & read and check & time & 216.958704\\\hline
proportional & EMFIncQuery-LocalSearch & 2 & SemaphoreNeighbor & read and check & time & 248.456908\\\hline
proportional & EMFIncQuery-Incremental & 4 & SemaphoreNeighbor & read and check & time & 411.017049\\\hline
proportional & EMFIncQuery-LocalSearch & 4 & SemaphoreNeighbor & read and check & time & 239.809039\\\hline
proportional & EMFIncQuery-Incremental & 8 & SemaphoreNeighbor & read and check & time & 253.538057\\\hline
proportional & EMFIncQuery-LocalSearch & 8 & SemaphoreNeighbor & read and check & time & 250.455944\\\hline
proportional & EMFIncQuery-Incremental & 16 & SemaphoreNeighbor & read and check & time & 445.688801\\\hline
proportional & EMFIncQuery-LocalSearch & 16 & SemaphoreNeighbor & read and check & time & 421.666751\\\hline
proportional & EMFIncQuery-Incremental & 32 & SemaphoreNeighbor & read and check & time & 1011.620549\\\hline
proportional & EMFIncQuery-LocalSearch & 32 & SemaphoreNeighbor & read and check & time & 947.223181\\\hline
proportional & EMFIncQuery-Incremental & 64 & SemaphoreNeighbor & read and check & time & 1955.677607\\\hline
proportional & EMFIncQuery-LocalSearch & 64 & SemaphoreNeighbor & read and check & time & 1649.115153\\\hline
proportional & EMFIncQuery-Incremental & 128 & SemaphoreNeighbor & read and check & time & 4406.712057\\\hline
proportional & EMFIncQuery-LocalSearch & 128 & SemaphoreNeighbor & read and check & time & 3281.632197\\\hline
proportional & EMFIncQuery-Incremental & 256 & SemaphoreNeighbor & read and check & time & 7763.485768\\\hline
proportional & EMFIncQuery-LocalSearch & 256 & SemaphoreNeighbor & read and check & time & 6390.738409\\\hline
proportional & EMFIncQuery-Incremental & 512 & SemaphoreNeighbor & read and check & time & 16694.718066\\\hline
proportional & EMFIncQuery-LocalSearch & 512 & SemaphoreNeighbor & read and check & time & 12879.359014\\\hline
proportional & EMFIncQuery-Incremental & 1024 & SemaphoreNeighbor & read and check & time & 43986.695158\\\hline
proportional & EMFIncQuery-LocalSearch & 1024 & SemaphoreNeighbor & read and check & time & 25950.281272\\\hline
proportional & EMFIncQuery-Incremental & 2048 & SemaphoreNeighbor & read and check & time & 96630.698725\\\hline
proportional & EMFIncQuery-LocalSearch & 2048 & SemaphoreNeighbor & read and check & time & 47487.288389\\\hline
proportional & EMFIncQuery-Incremental & 4096 & SemaphoreNeighbor & read and check & time & 235246.297145\\\hline
proportional & EMFIncQuery-LocalSearch & 4096 & SemaphoreNeighbor & read and check & time & 114993.525253\\\hline
proportional & EMFIncQuery-LocalSearch & 8192 & SemaphoreNeighbor & read and check & time & 274313.055853\\\hline

\end{tabular}\caption{Benchmark results for the \textsf{SemaphoreNeighbor} query.}
\label{tab:first-validation-semaphoreneighbor}
\end{table}

%\begin{table}
\centering
\footnotesize
\begin{tabular}{| l | l | r | l | l | l | r |}
\hline
\sf ChangeSet & \sf Tool & \sf Size & \sf Query & \sf PhaseName & \sf MetricName & \sf MetricValue\\\hline


fixed & EMFIncQuery-Incremental & 1 & SemaphoreNeighbor & modify and revalidate & time & 0.242763\\\hline
fixed & EMFIncQuery-LocalSearch & 1 & SemaphoreNeighbor & modify and revalidate & time & 48.919823\\\hline
fixed & EMFIncQuery-Incremental & 2 & SemaphoreNeighbor & modify and revalidate & time & 0.322082\\\hline
fixed & EMFIncQuery-LocalSearch & 2 & SemaphoreNeighbor & modify and revalidate & time & 67.607963\\\hline
fixed & EMFIncQuery-Incremental & 4 & SemaphoreNeighbor & modify and revalidate & time & 0.573263\\\hline
fixed & EMFIncQuery-LocalSearch & 4 & SemaphoreNeighbor & modify and revalidate & time & 144.420809\\\hline
fixed & EMFIncQuery-Incremental & 8 & SemaphoreNeighbor & modify and revalidate & time & 0.703411\\\hline
fixed & EMFIncQuery-LocalSearch & 8 & SemaphoreNeighbor & modify and revalidate & time & 238.202911\\\hline
fixed & EMFIncQuery-Incremental & 16 & SemaphoreNeighbor & modify and revalidate & time & 1.291134\\\hline
fixed & EMFIncQuery-LocalSearch & 16 & SemaphoreNeighbor & modify and revalidate & time & 405.1162\\\hline
fixed & EMFIncQuery-Incremental & 32 & SemaphoreNeighbor & modify and revalidate & time & 2.348602\\\hline
fixed & EMFIncQuery-LocalSearch & 32 & SemaphoreNeighbor & modify and revalidate & time & 917.987957\\\hline
fixed & EMFIncQuery-Incremental & 64 & SemaphoreNeighbor & modify and revalidate & time & 2.924311\\\hline
fixed & EMFIncQuery-LocalSearch & 64 & SemaphoreNeighbor & modify and revalidate & time & 1404.849753\\\hline
fixed & EMFIncQuery-Incremental & 128 & SemaphoreNeighbor & modify and revalidate & time & 2.979185\\\hline
fixed & EMFIncQuery-LocalSearch & 128 & SemaphoreNeighbor & modify and revalidate & time & 2803.220991\\\hline
fixed & EMFIncQuery-Incremental & 256 & SemaphoreNeighbor & modify and revalidate & time & 3.140119\\\hline
fixed & EMFIncQuery-LocalSearch & 256 & SemaphoreNeighbor & modify and revalidate & time & 5228.621765\\\hline
fixed & EMFIncQuery-Incremental & 512 & SemaphoreNeighbor & modify and revalidate & time & 3.100527\\\hline
fixed & EMFIncQuery-LocalSearch & 512 & SemaphoreNeighbor & modify and revalidate & time & 9532.445944\\\hline
fixed & EMFIncQuery-Incremental & 1024 & SemaphoreNeighbor & modify and revalidate & time & 3.184848\\\hline
fixed & EMFIncQuery-LocalSearch & 1024 & SemaphoreNeighbor & modify and revalidate & time & 19221.01202\\\hline
fixed & EMFIncQuery-Incremental & 2048 & SemaphoreNeighbor & modify and revalidate & time & 4.14882\\\hline
fixed & EMFIncQuery-LocalSearch & 2048 & SemaphoreNeighbor & modify and revalidate & time & 37380.322156\\\hline
fixed & EMFIncQuery-Incremental & 4096 & SemaphoreNeighbor & modify and revalidate & time & 4.092549\\\hline
fixed & EMFIncQuery-LocalSearch & 4096 & SemaphoreNeighbor & modify and revalidate & time & 74139.19707\\\hline
fixed & EMFIncQuery-LocalSearch & 8192 & SemaphoreNeighbor & modify and revalidate & time & 158821.741816\\\hline

proportional & EMFIncQuery-Incremental & 1 & SemaphoreNeighbor & modify and revalidate & time & 0.031402\\\hline
proportional & EMFIncQuery-LocalSearch & 1 & SemaphoreNeighbor & modify and revalidate & time & 50.690984\\\hline
proportional & EMFIncQuery-Incremental & 2 & SemaphoreNeighbor & modify and revalidate & time & 0.031876\\\hline
proportional & EMFIncQuery-LocalSearch & 2 & SemaphoreNeighbor & modify and revalidate & time & 83.068755\\\hline
proportional & EMFIncQuery-Incremental & 4 & SemaphoreNeighbor & modify and revalidate & time & 0.232779\\\hline
proportional & EMFIncQuery-LocalSearch & 4 & SemaphoreNeighbor & modify and revalidate & time & 149.420429\\\hline
proportional & EMFIncQuery-Incremental & 8 & SemaphoreNeighbor & modify and revalidate & time & 0.370724\\\hline
proportional & EMFIncQuery-LocalSearch & 8 & SemaphoreNeighbor & modify and revalidate & time & 241.393745\\\hline
proportional & EMFIncQuery-Incremental & 16 & SemaphoreNeighbor & modify and revalidate & time & 0.679953\\\hline
proportional & EMFIncQuery-LocalSearch & 16 & SemaphoreNeighbor & modify and revalidate & time & 417.64153\\\hline
proportional & EMFIncQuery-Incremental & 32 & SemaphoreNeighbor & modify and revalidate & time & 1.608991\\\hline
proportional & EMFIncQuery-LocalSearch & 32 & SemaphoreNeighbor & modify and revalidate & time & 917.727451\\\hline
proportional & EMFIncQuery-Incremental & 64 & SemaphoreNeighbor & modify and revalidate & time & 2.914975\\\hline
proportional & EMFIncQuery-LocalSearch & 64 & SemaphoreNeighbor & modify and revalidate & time & 1409.064017\\\hline
proportional & EMFIncQuery-Incremental & 128 & SemaphoreNeighbor & modify and revalidate & time & 6.909493\\\hline
proportional & EMFIncQuery-LocalSearch & 128 & SemaphoreNeighbor & modify and revalidate & time & 2788.790444\\\hline
proportional & EMFIncQuery-Incremental & 256 & SemaphoreNeighbor & modify and revalidate & time & 11.186405\\\hline
proportional & EMFIncQuery-LocalSearch & 256 & SemaphoreNeighbor & modify and revalidate & time & 5286.273787\\\hline
proportional & EMFIncQuery-Incremental & 512 & SemaphoreNeighbor & modify and revalidate & time & 21.959866\\\hline
proportional & EMFIncQuery-LocalSearch & 512 & SemaphoreNeighbor & modify and revalidate & time & 9707.476773\\\hline
proportional & EMFIncQuery-Incremental & 1024 & SemaphoreNeighbor & modify and revalidate & time & 38.594706\\\hline
proportional & EMFIncQuery-LocalSearch & 1024 & SemaphoreNeighbor & modify and revalidate & time & 18654.979515\\\hline
proportional & EMFIncQuery-Incremental & 2048 & SemaphoreNeighbor & modify and revalidate & time & 51.27372\\\hline
proportional & EMFIncQuery-LocalSearch & 2048 & SemaphoreNeighbor & modify and revalidate & time & 37016.909862\\\hline
proportional & EMFIncQuery-Incremental & 4096 & SemaphoreNeighbor & modify and revalidate & time & 47.797887\\\hline
proportional & EMFIncQuery-LocalSearch & 4096 & SemaphoreNeighbor & modify and revalidate & time & 70728.913349\\\hline
proportional & EMFIncQuery-LocalSearch & 8192 & SemaphoreNeighbor & modify and revalidate & time & 152640.781343\\\hline

\end{tabular}\caption{Benchmark results for the \textsf{SemaphoreNeighbor} query.}
\label{tab:revalidation-semaphoreneighbor}
\end{table}


%\begin{table}
\centering
\footnotesize
\begin{tabular}{| l | l | r | l | l | l | r |}
\hline
\sf ChangeSet & \sf Tool & \sf Size & \sf Query & \sf PhaseName & \sf MetricName & \sf MetricValue\\\hline
fixed & EMFIncQuery-Incremental & 1 & SwitchSensor & read and check & time & 212.145195\\\hline
fixed & EMFIncQuery-LocalSearch & 1 & SwitchSensor & read and check & time & 178.862836\\\hline
fixed & EMFIncQuery-Incremental & 2 & SwitchSensor & read and check & time & 224.406038\\\hline
fixed & EMFIncQuery-LocalSearch & 2 & SwitchSensor & read and check & time & 183.698027\\\hline
fixed & EMFIncQuery-Incremental & 4 & SwitchSensor & read and check & time & 433.539712\\\hline
fixed & EMFIncQuery-LocalSearch & 4 & SwitchSensor & read and check & time & 192.319116\\\hline
fixed & EMFIncQuery-Incremental & 8 & SwitchSensor & read and check & time & 616.233763\\\hline
fixed & EMFIncQuery-LocalSearch & 8 & SwitchSensor & read and check & time & 198.133643\\\hline
fixed & EMFIncQuery-Incremental & 16 & SwitchSensor & read and check & time & 378.528485\\\hline
fixed & EMFIncQuery-LocalSearch & 16 & SwitchSensor & read and check & time & 344.531458\\\hline
fixed & EMFIncQuery-Incremental & 32 & SwitchSensor & read and check & time & 786.93385\\\hline
fixed & EMFIncQuery-LocalSearch & 32 & SwitchSensor & read and check & time & 726.809429\\\hline
fixed & EMFIncQuery-Incremental & 64 & SwitchSensor & read and check & time & 1466.608066\\\hline
fixed & EMFIncQuery-LocalSearch & 64 & SwitchSensor & read and check & time & 1388.318983\\\hline
fixed & EMFIncQuery-Incremental & 128 & SwitchSensor & read and check & time & 2947.953852\\\hline
fixed & EMFIncQuery-LocalSearch & 128 & SwitchSensor & read and check & time & 2642.372399\\\hline
fixed & EMFIncQuery-Incremental & 256 & SwitchSensor & read and check & time & 5960.293711\\\hline
fixed & EMFIncQuery-LocalSearch & 256 & SwitchSensor & read and check & time & 5347.390309\\\hline
fixed & EMFIncQuery-Incremental & 512 & SwitchSensor & read and check & time & 11861.257029\\\hline
fixed & EMFIncQuery-LocalSearch & 512 & SwitchSensor & read and check & time & 11146.535654\\\hline
fixed & EMFIncQuery-Incremental & 1024 & SwitchSensor & read and check & time & 23584.506626\\\hline
fixed & EMFIncQuery-LocalSearch & 1024 & SwitchSensor & read and check & time & 22417.285269\\\hline
fixed & EMFIncQuery-Incremental & 2048 & SwitchSensor & read and check & time & 50169.889401\\\hline
fixed & EMFIncQuery-LocalSearch & 2048 & SwitchSensor & read and check & time & 42232.241293\\\hline
fixed & EMFIncQuery-Incremental & 4096 & SwitchSensor & read and check & time & 104605.912821\\\hline
fixed & EMFIncQuery-LocalSearch & 4096 & SwitchSensor & read and check & time & 79086.409073\\\hline
fixed & EMFIncQuery-Incremental & 8192 & SwitchSensor & read and check & time & 240151.226551\\\hline
fixed & EMFIncQuery-LocalSearch & 8192 & SwitchSensor & read and check & time & 209477.978433\\\hline


proportional & EMFIncQuery-Incremental & 1 & SwitchSensor & read and check & time & 213.801446\\\hline
proportional & EMFIncQuery-LocalSearch & 1 & SwitchSensor & read and check & time & 182.981747\\\hline
proportional & EMFIncQuery-Incremental & 2 & SwitchSensor & read and check & time & 225.278142\\\hline
proportional & EMFIncQuery-LocalSearch & 2 & SwitchSensor & read and check & time & 184.517582\\\hline
proportional & EMFIncQuery-Incremental & 4 & SwitchSensor & read and check & time & 411.685587\\\hline
proportional & EMFIncQuery-LocalSearch & 4 & SwitchSensor & read and check & time & 188.30868\\\hline
proportional & EMFIncQuery-Incremental & 8 & SwitchSensor & read and check & time & 652.210057\\\hline
proportional & EMFIncQuery-LocalSearch & 8 & SwitchSensor & read and check & time & 199.358978\\\hline
proportional & EMFIncQuery-Incremental & 16 & SwitchSensor & read and check & time & 350.374714\\\hline
proportional & EMFIncQuery-LocalSearch & 16 & SwitchSensor & read and check & time & 361.286201\\\hline
proportional & EMFIncQuery-Incremental & 32 & SwitchSensor & read and check & time & 761.357659\\\hline
proportional & EMFIncQuery-LocalSearch & 32 & SwitchSensor & read and check & time & 752.746563\\\hline
proportional & EMFIncQuery-Incremental & 64 & SwitchSensor & read and check & time & 1462.276392\\\hline
proportional & EMFIncQuery-LocalSearch & 64 & SwitchSensor & read and check & time & 1370.598727\\\hline
proportional & EMFIncQuery-Incremental & 128 & SwitchSensor & read and check & time & 3027.1606\\\hline
proportional & EMFIncQuery-LocalSearch & 128 & SwitchSensor & read and check & time & 2650.008175\\\hline
proportional & EMFIncQuery-Incremental & 256 & SwitchSensor & read and check & time & 6005.510775\\\hline
proportional & EMFIncQuery-LocalSearch & 256 & SwitchSensor & read and check & time & 5492.543772\\\hline
proportional & EMFIncQuery-Incremental & 512 & SwitchSensor & read and check & time & 11762.286475\\\hline
proportional & EMFIncQuery-LocalSearch & 512 & SwitchSensor & read and check & time & 10692.195111\\\hline
proportional & EMFIncQuery-Incremental & 1024 & SwitchSensor & read and check & time & 23570.287008\\\hline
proportional & EMFIncQuery-LocalSearch & 1024 & SwitchSensor & read and check & time & 21643.955781\\\hline
proportional & EMFIncQuery-Incremental & 2048 & SwitchSensor & read and check & time & 49683.130107\\\hline
proportional & EMFIncQuery-LocalSearch & 2048 & SwitchSensor & read and check & time & 43369.487551\\\hline
proportional & EMFIncQuery-Incremental & 4096 & SwitchSensor & read and check & time & 102994.540092\\\hline
proportional & EMFIncQuery-LocalSearch & 4096 & SwitchSensor & read and check & time & 78256.129946\\\hline
proportional & EMFIncQuery-Incremental & 8192 & SwitchSensor & read and check & time & 234484.572557\\\hline
proportional & EMFIncQuery-LocalSearch & 8192 & SwitchSensor & read and check & time & 204665.946322\\\hline

\end{tabular}\caption{Benchmark results for the \textsf{SwitchSensor} query.}
\label{tab:first-validation-switchsensor}
\end{table}

%\begin{table}
\centering
\footnotesize
\begin{tabular}{| l | l | r | l | l | l | r |}
\hline
\sf ChangeSet & \sf Tool & \sf Size & \sf Query & \sf PhaseName & \sf MetricName & \sf MetricValue\\\hline


fixed & EMFIncQuery-Incremental & 1 & SwitchSensor & modify and revalidate & time & 0.465696\\\hline
fixed & EMFIncQuery-LocalSearch & 1 & SwitchSensor & modify and revalidate & time & 38.761292\\\hline
fixed & EMFIncQuery-Incremental & 2 & SwitchSensor & modify and revalidate & time & 0.876356\\\hline
fixed & EMFIncQuery-LocalSearch & 2 & SwitchSensor & modify and revalidate & time & 49.098053\\\hline
fixed & EMFIncQuery-Incremental & 4 & SwitchSensor & modify and revalidate & time & 1.800714\\\hline
fixed & EMFIncQuery-LocalSearch & 4 & SwitchSensor & modify and revalidate & time & 100.187725\\\hline
fixed & EMFIncQuery-Incremental & 8 & SwitchSensor & modify and revalidate & time & 3.061915\\\hline
fixed & EMFIncQuery-LocalSearch & 8 & SwitchSensor & modify and revalidate & time & 22.73719\\\hline
fixed & EMFIncQuery-Incremental & 16 & SwitchSensor & modify and revalidate & time & 4.022412\\\hline
fixed & EMFIncQuery-LocalSearch & 16 & SwitchSensor & modify and revalidate & time & 29.137434\\\hline
fixed & EMFIncQuery-Incremental & 32 & SwitchSensor & modify and revalidate & time & 5.755359\\\hline
fixed & EMFIncQuery-LocalSearch & 32 & SwitchSensor & modify and revalidate & time & 54.47707\\\hline
fixed & EMFIncQuery-Incremental & 64 & SwitchSensor & modify and revalidate & time & 6.014394\\\hline
fixed & EMFIncQuery-LocalSearch & 64 & SwitchSensor & modify and revalidate & time & 93.700505\\\hline
fixed & EMFIncQuery-Incremental & 128 & SwitchSensor & modify and revalidate & time & 5.864815\\\hline
fixed & EMFIncQuery-LocalSearch & 128 & SwitchSensor & modify and revalidate & time & 163.466406\\\hline
fixed & EMFIncQuery-Incremental & 256 & SwitchSensor & modify and revalidate & time & 5.760812\\\hline
fixed & EMFIncQuery-LocalSearch & 256 & SwitchSensor & modify and revalidate & time & 301.866686\\\hline
fixed & EMFIncQuery-Incremental & 512 & SwitchSensor & modify and revalidate & time & 6.344689\\\hline
fixed & EMFIncQuery-LocalSearch & 512 & SwitchSensor & modify and revalidate & time & 536.210619\\\hline
fixed & EMFIncQuery-Incremental & 1024 & SwitchSensor & modify and revalidate & time & 7.754294\\\hline
fixed & EMFIncQuery-LocalSearch & 1024 & SwitchSensor & modify and revalidate & time & 1010.215018\\\hline
fixed & EMFIncQuery-Incremental & 2048 & SwitchSensor & modify and revalidate & time & 10.409721\\\hline
fixed & EMFIncQuery-LocalSearch & 2048 & SwitchSensor & modify and revalidate & time & 1961.527318\\\hline
fixed & EMFIncQuery-Incremental & 4096 & SwitchSensor & modify and revalidate & time & 14.983758\\\hline
fixed & EMFIncQuery-LocalSearch & 4096 & SwitchSensor & modify and revalidate & time & 4850.828713\\\hline
fixed & EMFIncQuery-Incremental & 8192 & SwitchSensor & modify and revalidate & time & 29.774253\\\hline
fixed & EMFIncQuery-LocalSearch & 8192 & SwitchSensor & modify and revalidate & time & 7956.81605\\\hline


proportional & EMFIncQuery-Incremental & 1 & SwitchSensor & modify and revalidate & time & 0.04129\\\hline
proportional & EMFIncQuery-LocalSearch & 1 & SwitchSensor & modify and revalidate & time & 37.971389\\\hline
proportional & EMFIncQuery-Incremental & 2 & SwitchSensor & modify and revalidate & time & 0.034927\\\hline
proportional & EMFIncQuery-LocalSearch & 2 & SwitchSensor & modify and revalidate & time & 54.251865\\\hline
proportional & EMFIncQuery-Incremental & 4 & SwitchSensor & modify and revalidate & time & 1.040301\\\hline
proportional & EMFIncQuery-LocalSearch & 4 & SwitchSensor & modify and revalidate & time & 109.519572\\\hline
proportional & EMFIncQuery-Incremental & 8 & SwitchSensor & modify and revalidate & time & 1.509935\\\hline
proportional & EMFIncQuery-LocalSearch & 8 & SwitchSensor & modify and revalidate & time & 24.404587\\\hline
proportional & EMFIncQuery-Incremental & 16 & SwitchSensor & modify and revalidate & time & 2.602653\\\hline
proportional & EMFIncQuery-LocalSearch & 16 & SwitchSensor & modify and revalidate & time & 30.023173\\\hline
proportional & EMFIncQuery-Incremental & 32 & SwitchSensor & modify and revalidate & time & 4.529747\\\hline
proportional & EMFIncQuery-LocalSearch & 32 & SwitchSensor & modify and revalidate & time & 56.497694\\\hline
proportional & EMFIncQuery-Incremental & 64 & SwitchSensor & modify and revalidate & time & 8.45335\\\hline
proportional & EMFIncQuery-LocalSearch & 64 & SwitchSensor & modify and revalidate & time & 93.539656\\\hline
proportional & EMFIncQuery-Incremental & 128 & SwitchSensor & modify and revalidate & time & 16.496139\\\hline
proportional & EMFIncQuery-LocalSearch & 128 & SwitchSensor & modify and revalidate & time & 182.149298\\\hline
proportional & EMFIncQuery-Incremental & 256 & SwitchSensor & modify and revalidate & time & 31.732373\\\hline
proportional & EMFIncQuery-LocalSearch & 256 & SwitchSensor & modify and revalidate & time & 318.135602\\\hline
proportional & EMFIncQuery-Incremental & 512 & SwitchSensor & modify and revalidate & time & 64.42754\\\hline
proportional & EMFIncQuery-LocalSearch & 512 & SwitchSensor & modify and revalidate & time & 547.279917\\\hline
proportional & EMFIncQuery-Incremental & 1024 & SwitchSensor & modify and revalidate & time & 136.801998\\\hline
proportional & EMFIncQuery-LocalSearch & 1024 & SwitchSensor & modify and revalidate & time & 1117.288917\\\hline
proportional & EMFIncQuery-Incremental & 2048 & SwitchSensor & modify and revalidate & time & 317.503777\\\hline
proportional & EMFIncQuery-LocalSearch & 2048 & SwitchSensor & modify and revalidate & time & 2128.361921\\\hline
proportional & EMFIncQuery-Incremental & 4096 & SwitchSensor & modify and revalidate & time & 1067.108193\\\hline
proportional & EMFIncQuery-LocalSearch & 4096 & SwitchSensor & modify and revalidate & time & 5626.040347\\\hline
proportional & EMFIncQuery-Incremental & 8192 & SwitchSensor & modify and revalidate & time & 4231.575623\\\hline
proportional & EMFIncQuery-LocalSearch & 8192 & SwitchSensor & modify and revalidate & time & 11322.67485\\\hline

\end{tabular}\caption{Benchmark results for the \textsf{SwitchSensor} query.}
\label{tab:revalidation-switchsensor}
\end{table}


%\begin{table}
\centering
\footnotesize
\begin{tabular}{| l | l | r | l | l | l | r |}
\hline
\sf ChangeSet & \sf Tool & \sf Size & \sf Query & \sf PhaseName & \sf MetricName & \sf MetricValue\\\hline

fixed & EMFIncQuery-Incremental & 1 & SwitchSet & read and check & time & 208.286552\\\hline
fixed & EMFIncQuery-LocalSearch & 1 & SwitchSet & read and check & time & 178.641895\\\hline
fixed & EMFIncQuery-Incremental & 2 & SwitchSet & read and check & time & 232.598123\\\hline
fixed & EMFIncQuery-LocalSearch & 2 & SwitchSet & read and check & time & 193.606064\\\hline
fixed & EMFIncQuery-Incremental & 4 & SwitchSet & read and check & time & 435.633688\\\hline
fixed & EMFIncQuery-LocalSearch & 4 & SwitchSet & read and check & time & 344.405117\\\hline
fixed & EMFIncQuery-Incremental & 8 & SwitchSet & read and check & time & 526.357238\\\hline
fixed & EMFIncQuery-LocalSearch & 8 & SwitchSet & read and check & time & 393.372673\\\hline
fixed & EMFIncQuery-Incremental & 16 & SwitchSet & read and check & time & 367.026443\\\hline
fixed & EMFIncQuery-LocalSearch & 16 & SwitchSet & read and check & time & 335.331063\\\hline
fixed & EMFIncQuery-Incremental & 32 & SwitchSet & read and check & time & 742.550248\\\hline
fixed & EMFIncQuery-LocalSearch & 32 & SwitchSet & read and check & time & 745.474537\\\hline
fixed & EMFIncQuery-Incremental & 64 & SwitchSet & read and check & time & 1409.785411\\\hline
fixed & EMFIncQuery-LocalSearch & 64 & SwitchSet & read and check & time & 1337.410146\\\hline
fixed & EMFIncQuery-Incremental & 128 & SwitchSet & read and check & time & 2723.686899\\\hline
fixed & EMFIncQuery-LocalSearch & 128 & SwitchSet & read and check & time & 2654.318845\\\hline
fixed & EMFIncQuery-Incremental & 256 & SwitchSet & read and check & time & 5228.220024\\\hline
fixed & EMFIncQuery-LocalSearch & 256 & SwitchSet & read and check & time & 5109.74452\\\hline
fixed & EMFIncQuery-Incremental & 512 & SwitchSet & read and check & time & 10647.262411\\\hline
fixed & EMFIncQuery-LocalSearch & 512 & SwitchSet & read and check & time & 10255.406911\\\hline
fixed & EMFIncQuery-Incremental & 1024 & SwitchSet & read and check & time & 21832.349604\\\hline
fixed & EMFIncQuery-LocalSearch & 1024 & SwitchSet & read and check & time & 21812.854842\\\hline
fixed & EMFIncQuery-Incremental & 2048 & SwitchSet & read and check & time & 43964.996234\\\hline
fixed & EMFIncQuery-LocalSearch & 2048 & SwitchSet & read and check & time & 42751.28656\\\hline
fixed & EMFIncQuery-Incremental & 4096 & SwitchSet & read and check & time & 86172.978543\\\hline
fixed & EMFIncQuery-LocalSearch & 4096 & SwitchSet & read and check & time & 81218.836356\\\hline
fixed & EMFIncQuery-Incremental & 8192 & SwitchSet & read and check & time & 214932.335954\\\hline
fixed & EMFIncQuery-LocalSearch & 8192 & SwitchSet & read and check & time & 205799.651229\\\hline


proportional & EMFIncQuery-Incremental & 1 & SwitchSet & read and check & time & 202.849318\\\hline
proportional & EMFIncQuery-LocalSearch & 1 & SwitchSet & read and check & time & 173.902516\\\hline
proportional & EMFIncQuery-Incremental & 2 & SwitchSet & read and check & time & 236.147014\\\hline
proportional & EMFIncQuery-LocalSearch & 2 & SwitchSet & read and check & time & 202.472323\\\hline
proportional & EMFIncQuery-Incremental & 4 & SwitchSet & read and check & time & 429.291008\\\hline
proportional & EMFIncQuery-LocalSearch & 4 & SwitchSet & read and check & time & 353.470199\\\hline
proportional & EMFIncQuery-Incremental & 8 & SwitchSet & read and check & time & 448.922723\\\hline
proportional & EMFIncQuery-LocalSearch & 8 & SwitchSet & read and check & time & 296.550607\\\hline
proportional & EMFIncQuery-Incremental & 16 & SwitchSet & read and check & time & 377.83214\\\hline
proportional & EMFIncQuery-LocalSearch & 16 & SwitchSet & read and check & time & 337.867756\\\hline
proportional & EMFIncQuery-Incremental & 32 & SwitchSet & read and check & time & 733.572509\\\hline
proportional & EMFIncQuery-LocalSearch & 32 & SwitchSet & read and check & time & 767.643292\\\hline
proportional & EMFIncQuery-Incremental & 64 & SwitchSet & read and check & time & 1385.439209\\\hline
proportional & EMFIncQuery-LocalSearch & 64 & SwitchSet & read and check & time & 1342.553532\\\hline
proportional & EMFIncQuery-Incremental & 128 & SwitchSet & read and check & time & 2735.031563\\\hline
proportional & EMFIncQuery-LocalSearch & 128 & SwitchSet & read and check & time & 2653.103857\\\hline
proportional & EMFIncQuery-Incremental & 256 & SwitchSet & read and check & time & 5211.693255\\\hline
proportional & EMFIncQuery-LocalSearch & 256 & SwitchSet & read and check & time & 5121.656312\\\hline
proportional & EMFIncQuery-Incremental & 512 & SwitchSet & read and check & time & 10575.249589\\\hline
proportional & EMFIncQuery-LocalSearch & 512 & SwitchSet & read and check & time & 10265.694798\\\hline
proportional & EMFIncQuery-Incremental & 1024 & SwitchSet & read and check & time & 21943.124502\\\hline
proportional & EMFIncQuery-LocalSearch & 1024 & SwitchSet & read and check & time & 21939.726695\\\hline
proportional & EMFIncQuery-Incremental & 2048 & SwitchSet & read and check & time & 47219.301477\\\hline
proportional & EMFIncQuery-LocalSearch & 2048 & SwitchSet & read and check & time & 43977.344409\\\hline
proportional & EMFIncQuery-Incremental & 4096 & SwitchSet & read and check & time & 85771.758384\\\hline
proportional & EMFIncQuery-LocalSearch & 4096 & SwitchSet & read and check & time & 82030.447963\\\hline
proportional & EMFIncQuery-Incremental & 8192 & SwitchSet & read and check & time & 213949.725008\\\hline
proportional & EMFIncQuery-LocalSearch & 8192 & SwitchSet & read and check & time & 208860.208973\\\hline

\end{tabular}\caption{Benchmark results for the \textsf{SwitchSet} query.}
\label{tab:first-validation-switchset}
\end{table}

%\begin{table}
\centering
\footnotesize
\begin{tabular}{| l | l | r | l | l | l | r |}
\hline
\sf ChangeSet & \sf Tool & \sf Size & \sf Query & \sf PhaseName & \sf MetricName & \sf MetricValue\\\hline


fixed & EMFIncQuery-Incremental & 1 & SwitchSet & modify and revalidate & time & 0.923453\\\hline
fixed & EMFIncQuery-LocalSearch & 1 & SwitchSet & modify and revalidate & time & 24.335329\\\hline
fixed & EMFIncQuery-Incremental & 2 & SwitchSet & modify and revalidate & time & 0.66494\\\hline
fixed & EMFIncQuery-LocalSearch & 2 & SwitchSet & modify and revalidate & time & 29.480884\\\hline
fixed & EMFIncQuery-Incremental & 4 & SwitchSet & modify and revalidate & time & 3.230078\\\hline
fixed & EMFIncQuery-LocalSearch & 4 & SwitchSet & modify and revalidate & time & 28.635818\\\hline
fixed & EMFIncQuery-Incremental & 8 & SwitchSet & modify and revalidate & time & 6.425558\\\hline
fixed & EMFIncQuery-LocalSearch & 8 & SwitchSet & modify and revalidate & time & 37.231442\\\hline
fixed & EMFIncQuery-Incremental & 16 & SwitchSet & modify and revalidate & time & 5.890825\\\hline
fixed & EMFIncQuery-LocalSearch & 16 & SwitchSet & modify and revalidate & time & 30.382119\\\hline
fixed & EMFIncQuery-Incremental & 32 & SwitchSet & modify and revalidate & time & 5.277236\\\hline
fixed & EMFIncQuery-LocalSearch & 32 & SwitchSet & modify and revalidate & time & 42.026298\\\hline
fixed & EMFIncQuery-Incremental & 64 & SwitchSet & modify and revalidate & time & 5.035648\\\hline
fixed & EMFIncQuery-LocalSearch & 64 & SwitchSet & modify and revalidate & time & 51.385197\\\hline
fixed & EMFIncQuery-Incremental & 128 & SwitchSet & modify and revalidate & time & 5.669557\\\hline
fixed & EMFIncQuery-LocalSearch & 128 & SwitchSet & modify and revalidate & time & 54.736698\\\hline
fixed & EMFIncQuery-Incremental & 256 & SwitchSet & modify and revalidate & time & 5.674425\\\hline
fixed & EMFIncQuery-LocalSearch & 256 & SwitchSet & modify and revalidate & time & 92.260942\\\hline
fixed & EMFIncQuery-Incremental & 512 & SwitchSet & modify and revalidate & time & 4.836251\\\hline
fixed & EMFIncQuery-LocalSearch & 512 & SwitchSet & modify and revalidate & time & 133.983525\\\hline
fixed & EMFIncQuery-Incremental & 1024 & SwitchSet & modify and revalidate & time & 5.741573\\\hline
fixed & EMFIncQuery-LocalSearch & 1024 & SwitchSet & modify and revalidate & time & 266.665015\\\hline
fixed & EMFIncQuery-Incremental & 2048 & SwitchSet & modify and revalidate & time & 6.29377\\\hline
fixed & EMFIncQuery-LocalSearch & 2048 & SwitchSet & modify and revalidate & time & 502.177003\\\hline
fixed & EMFIncQuery-Incremental & 4096 & SwitchSet & modify and revalidate & time & 6.121537\\\hline
fixed & EMFIncQuery-LocalSearch & 4096 & SwitchSet & modify and revalidate & time & 991.881967\\\hline
fixed & EMFIncQuery-Incremental & 8192 & SwitchSet & modify and revalidate & time & 6.838978\\\hline
fixed & EMFIncQuery-LocalSearch & 8192 & SwitchSet & modify and revalidate & time & 2893.51237\\\hline

proportional & EMFIncQuery-Incremental & 1 & SwitchSet & modify and revalidate & time & 0.037554\\\hline
proportional & EMFIncQuery-LocalSearch & 1 & SwitchSet & modify and revalidate & time & 24.544257\\\hline
proportional & EMFIncQuery-Incremental & 2 & SwitchSet & modify and revalidate & time & 0.040755\\\hline
proportional & EMFIncQuery-LocalSearch & 2 & SwitchSet & modify and revalidate & time & 32.413495\\\hline
proportional & EMFIncQuery-Incremental & 4 & SwitchSet & modify and revalidate & time & 0.854157\\\hline
proportional & EMFIncQuery-LocalSearch & 4 & SwitchSet & modify and revalidate & time & 29.179642\\\hline
proportional & EMFIncQuery-Incremental & 8 & SwitchSet & modify and revalidate & time & 3.498142\\\hline
proportional & EMFIncQuery-LocalSearch & 8 & SwitchSet & modify and revalidate & time & 41.66948\\\hline
proportional & EMFIncQuery-Incremental & 16 & SwitchSet & modify and revalidate & time & 4.04174\\\hline
proportional & EMFIncQuery-LocalSearch & 16 & SwitchSet & modify and revalidate & time & 33.742092\\\hline
proportional & EMFIncQuery-Incremental & 32 & SwitchSet & modify and revalidate & time & 5.538361\\\hline
proportional & EMFIncQuery-LocalSearch & 32 & SwitchSet & modify and revalidate & time & 42.779298\\\hline
proportional & EMFIncQuery-Incremental & 64 & SwitchSet & modify and revalidate & time & 10.647013\\\hline
proportional & EMFIncQuery-LocalSearch & 64 & SwitchSet & modify and revalidate & time & 58.024174\\\hline
proportional & EMFIncQuery-Incremental & 128 & SwitchSet & modify and revalidate & time & 19.033587\\\hline
proportional & EMFIncQuery-LocalSearch & 128 & SwitchSet & modify and revalidate & time & 57.20849\\\hline
proportional & EMFIncQuery-Incremental & 256 & SwitchSet & modify and revalidate & time & 34.238173\\\hline
proportional & EMFIncQuery-LocalSearch & 256 & SwitchSet & modify and revalidate & time & 96.377388\\\hline
proportional & EMFIncQuery-Incremental & 512 & SwitchSet & modify and revalidate & time & 47.63571\\\hline
proportional & EMFIncQuery-LocalSearch & 512 & SwitchSet & modify and revalidate & time & 138.035385\\\hline
proportional & EMFIncQuery-Incremental & 1024 & SwitchSet & modify and revalidate & time & 73.157516\\\hline
proportional & EMFIncQuery-LocalSearch & 1024 & SwitchSet & modify and revalidate & time & 237.37585\\\hline
proportional & EMFIncQuery-Incremental & 2048 & SwitchSet & modify and revalidate & time & 78.99769\\\hline
proportional & EMFIncQuery-LocalSearch & 2048 & SwitchSet & modify and revalidate & time & 438.623374\\\hline
proportional & EMFIncQuery-Incremental & 4096 & SwitchSet & modify and revalidate & time & 175.037324\\\hline
proportional & EMFIncQuery-LocalSearch & 4096 & SwitchSet & modify and revalidate & time & 876.239088\\\hline
proportional & EMFIncQuery-Incremental & 8192 & SwitchSet & modify and revalidate & time & 324.747253\\\hline
proportional & EMFIncQuery-LocalSearch & 8192 & SwitchSet & modify and revalidate & time & 2066.011621\\\hline

\end{tabular}\caption{Benchmark results for the \textsf{SwitchSet} query.}
\label{tab:revalidation-switchset}
\end{table}


\clearpage

\subsection{Rete Network}

\ttcfig{rete-routesensor-layout}{The Rete network for the \textsf{RouteSensor} query}{0.7}

\FloatBarrier

\subsection{Local Search Plan}

\ttcfig{searchplan-routesensor}{The search plan and the matches for the \textsf{RouteSensor} query}{0.6}

\end{document}
